\documentclass[]{article}

\begin{document}


\section{Richtungslichtquelle}

Die endgültige Lichtintensität, die das Auge erreicht setzt sich zusammen aus: \\
 
$I = I_E + I_B + I_L$

dem emittierten Licht:

$I_E = \int\limits_{t_n}^{t_f} \tau(t_n, t) \cdot c(v_r(t)) dt $

dem ambienten Licht:

$I_B = \tau(t_n, t_f) \cdot I_b$

$\tau(t_0, t_1) = e^{- \int\limits_{t_0}^{t_1} \sigma (v_r(t)) dt}$ \\

und dem Licht der Einzelnen Lichtquellen. Hier reduziert auf einen einzelnen Strahl der den Sichtstrahl $\underline{r}(t) = \underline{e} + t \cdot \vec{v}$ im Volumen schneidet. (An der Stelle t)

$I_L(t) = \tau(t_n, t) \cdot I_p(t)$

$I_p(t) = e^{- \int\limits_{0}^{l(t)} \cdot  \sigma (v_r(s)) ds} \cdot I_p$

$I_p(t) = e^{- \int\limits_{0}^{l(t)} \cdot  \lambda \cdot v_r(s) ds} \cdot I_p$

$I_p(t) = e^{- \int\limits_{0}^{l(t)} \cdot  \lambda \cdot \kappa ds} \cdot I_p$

$I_p(t) = e^{- \int\limits_{0}^{l(t)} \cdot  \lambda \cdot \kappa ds} \cdot I_p$

$I_p(t) = e^{- \lambda \cdot \kappa \cdot \int\limits_{0}^{l(t)} ds} \cdot I_p$

$I_p(t) = e^{- \lambda \cdot \kappa \cdot s} \cdot I_p$

$I_p(t) = e^{- \lambda \cdot \kappa \cdot \frac{-<p(t), d(t)>}{<d(t), d(t)>} + \frac{\sqrt{<p(t), d(t)>^2 - <p(t), p(t)> \cdot <d(t), d(t)>}}{<d(t), d(t)>}} \cdot I_p$ \\

mit $d(t) = d$ da es sich um eine Richtungslichtquelle handelt \\

$I_p(t) = e^{- \lambda \cdot \kappa \cdot \frac{-<p(t), d>}{<d, d>} + \frac{\sqrt{<p(t), d>^2 - <p(t), p(t)> \cdot <d, d>}}{<d, d>}} \cdot I_p$

$I_{L(t)} = \tau(t_n, t) \cdot e^{- \lambda \cdot \kappa \cdot \frac{-<p(t), d>}{<d, d>} + \frac{\sqrt{<p(t), d>^2 - <p(t), p(t)> \cdot <d, d>}}{<d, d>}} \cdot I_p$ \\

Daraus ergibt sich $I_L$ für eine Richtungslichtquelle \\

$I_L = \int\limits_{t_n}^{t_f} I_{L(t)} dt $

$I_L = \int\limits_{t_n}^{t_f} \tau(t_n, t) \cdot I_{p(t)} dt$

$I_L = \int\limits_{t_n}^{t_f} \tau(t_n, t) \cdot e^{- \lambda \cdot \kappa \cdot \frac{-<p(t), d>}{<d, d>} + \frac{\sqrt{<p(t), d>^2 - <p(t), p(t)> \cdot <d, d>}}{<d, d>}} \cdot I_p dt$

\end{document}
