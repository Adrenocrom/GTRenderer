\documentclass[hyperref,german,beleg]{cgvpub}
%weitere Optionen zum Erg�nzen (in eckigen Klammern):
% 
% female	weibliche Titelbezeichnung bei Diplom
% bibnum	numerische Literaturschl�ssel
% final 	f�r Abgabe	
% lof			Abbildungsverzeichis
% lot			Tabellenverzeichnis
% noproblem	keine Aufgabenstellung
% notoc			kein Inhaltsverzeichnis
% twoside		zweiseitig
\author{Josef Schulz}
\title{Ground-Truth-Renderer f�r Partikelbasierte Daten}
\birthday{20. Oktober 1989}
\placeofbirth{Dresden}
\matno{3658867}

\betreuer{Dipl-MedienInf. Joachim Staib}
\bibfiles{literatur}
\problem{Text der Aufgabenstellung...}
\copyrighterklaerung{Hier soll jeder Autor die von ihm eingeholten
Zustimmungen der Copyright-Besitzer angeben bzw. die in Web Press
Rooms angegebenen generellen Konditionen seiner Text- und
Bild"ubernahmen zitieren.}
\acknowledgments{Die Danksagung...}

\begin{document}
\chapter{Einleitung}
\section{Vorwort}
	Die Entwicklung des ersten Funktionsf�higen Ray-Tracers wird auf  Arthur Appel, Robert Goldstein und Roger Nagel zur�ckgef�hrt. 
	Diese erzeugten im Jahr 1963 an der University of Maryland, auf einem oszilloskopartigen Bildschirm das erste mit Ray-Tracing berechnete
	Bild. Viele der Verwendeten Algorithmen worden jedoch schon Jahre zuvor entwickelt. Nach Jahren der Weiterentwicklung von Computern,
	stehen heute Leistungsf�hige Rechenmaschinen zur Verf�gung die den Einsatz von Ray-Tracern praktikabel machen.
	
	Ein Ray-Tracer ist ein Programm, welches die Wege und die Ausbreitung des Lichtes simuliert. Hierbei kann der Weg des Lichtes, entweder
	ausgehend von der Lichtquelle, oder ausgehend von der Kamera, der Bildebene nachvollzogen werden.
	Betrachtet man den zweiten Fall, wird das Bild in Pixel unterteilt. Von jeden Pixel werden Strahlen in die Szene geschossen. An den Stellen,
	wo diese auf Objekte in der Szene treffen, wird eine Beleuchtungsrechnung durchgef�hrt. Die ermittelten Lichtintensit�ten werden akkumuliert und
	in Farben �bersetzt. Auf diese Weise lassen sich f�r jeden Pixel des Bildes die Farbwerte bestimmen.
	
	Mit einem Ray-Tracer, lassen sich Szenen nach realen physikalischen Modellen beleuchten. Aus diesem Grund ist es ein Verfahren, mit dessen Hilfe
	sich Ground-Truth Daten erzeugen lassen. Ground-Truth bezeichnet in diesem Falle eine Optimale L�sung, die als Vergleichswert zur Evaluation
	anderer Verfahren geeignet ist.
	
	Im Rahmen dieser Arbeit, wird der Aufbau eines Ray-Tracers beschrieben. Ben�tigt wird dieser zur realistischen Beleuchtung von Kugelglyphen.
	Dabei wird aus Partikeldaten eine Dichtefunktion gesch�tzt, welche die Materialeigenschaften der Kugel an jedem Punkt im Volumen definiert.
	Der Ray-Tracer, wird hierzu nach und nach erweitert. Zu beginn wird die Beleuchtungsrechnung ausschlie�lich auf der Oberfl�che durchgef�hrt.
	Diese wird sp�ter auf Volumen erweitert. Ziel dieser Arbeit ist es die Rendergleichung, m�glichst Analytisch zu l�sen.

\section{Vorbetrachtungen}

\chapter{Datens�tze}
\chapter{Rendergleichung}
\chapter{Ray-Tracing}
\chapter{Evaluation}

%\chapter{title}
%
%\chapter{ein kapitel}
%\section{eine Grafik}
%\begin{figure}[htbp]
%	\centering
%		\includegraphics{test.png}
%	\caption{beschriftung}
%	\label{fig:diplominf}
%\end{figure}


%\subsection{Etwas Mathe}
%
%\[
%\sum_{i=1}^{100}x_i
%\]
%noch mehr text
%\subsubsection{Verweise auf Literatur}
%So kann ich Literatur aus literatur.bib zitieren: \cite{kochbuch}. 
%
%\paragraph{etwas quelltext}
%
%
%\begin{figure}[htbp]
%\begin{lstlisting}[frame=trbl]
%//comment
%for(int i = 0; i < 100;i++)
%{
%test(i);
%}
%\end{lstlisting}
%\end{figure}
%
%text
%
%\cite*{}
\end{document}